@inproceedings{10.1145/1341012.1341055,
author = {Johansson, Sara and Jern, Mikael},
title = {GeoAnalytics Visual Inquiry and Filtering Tools in Parallel Coordinates Plots},
year = {2007},
isbn = {9781595939142},
publisher = {Association for Computing Machinery},
address = {New York, NY, USA},
url = {https://doi.org/10.1145/1341012.1341055},
doi = {10.1145/1341012.1341055},
abstract = {The complex nature of social and scientific spatial-temporal multivariate data calls for highly interactive integrated information visualization (InfoVis) and geo-visualization (GeoVis) tools and applications. Our research concentrates on improving visual user interface (VUI) methods and extending existing visual representation techniques.In this paper, we introduce an enhanced parallel coordinates (PC) component, integrating statistical methods for visual inquiries and filtering of spatial and multivariate data, a component that provides a fast and intuitive understanding of the distribution of data and of the relationships between data attributes. This enhanced PC component can also serve as a visual control panel for dynamically steering applications with multiple-linked and coordinated InfoVis and GeoVis views.The effectiveness of our proposed extended PC component is demonstrated in a tailor-made application called GeoWizard Lite which is based on our 'GeoAnalytics' visualization (GAV) framework. The GAV framework is based on the principles behind Visual Analytics (VA) and the developement is aiming to facilitate the extraction of complex patterns in large data sets via visual interaction. In this paper we also demonstrate the synergy between InfoVis and GeoVis methods through this case study exploring social science data from the Statistics Sweden data bases.},
booktitle = {Proceedings of the 15th Annual ACM International Symposium on Advances in Geographic Information Systems},
articleno = {33},
numpages = {8},
keywords = {interaction, visual user interface, information visualization, parallel coordinates, visual inquiries, geo-visualization},
location = {Seattle, Washington},
series = {GIS '07}
}