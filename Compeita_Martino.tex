@article{COMPIETA2007255,
title = {Exploratory spatio-temporal data mining and visualization},
journal = {Journal of Visual Languages & Computing},
volume = {18},
number = {3},
pages = {255-279},
year = {2007},
note = {Visual Languages and Techniques for Human-GIS Interaction},
issn = {1045-926X},
doi = {https://doi.org/10.1016/j.jvlc.2007.02.006},
url = {https://www.sciencedirect.com/science/article/pii/S1045926X07000134},
author = {P. Compieta and S. {Di Martino} and M. Bertolotto and F. Ferrucci and T. Kechadi},
keywords = {Data mining, Spatio-temporal data, Exploratory visualization},
abstract = {Spatio-temporal data sets are often very large and difficult to analyze and display. Since they are fundamental for decision support in many application contexts, recently a lot of interest has arisen toward data-mining techniques to filter out relevant subsets of very large data repositories as well as visualization tools to effectively display the results. In this paper we propose a data-mining system to deal with very large spatio-temporal data sets. Within this system, new techniques have been developed to efficiently support the data-mining process, address the spatial and temporal dimensions of the data set, and visualize and interpret results. In particular, two complementary 3D visualization environments have been implemented. One exploits Google Earth to display the mining outcomes combined with a map and other geographical layers, while the other is a Java3D-based tool for providing advanced interactions with the data set in a non-geo-referenced space, such as displaying association rules and variable distributions.}
}