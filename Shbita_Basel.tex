@InProceedings{10.1007/978-3-030-49461-2_24,
author="Shbita, Basel
and Knoblock, Craig A.
and Duan, Weiwei
and Chiang, Yao-Yi
and Uhl, Johannes H.
and Leyk, Stefan",
editor="Harth, Andreas
and Kirrane, Sabrina
and Ngonga Ngomo, Axel-Cyrille
and Paulheim, Heiko
and Rula, Anisa
and Gentile, Anna Lisa
and Haase, Peter
and Cochez, Michael",
title="Building Linked Spatio-Temporal Data from Vectorized Historical Maps",
booktitle="The Semantic Web",
year="2020",
publisher="Springer International Publishing",
address="Cham",
pages="409--426",
abstract="Historical maps provide a rich source of information for researchers in the social and natural sciences. These maps contain detailed documentation of a wide variety of natural and human-made features and their changes over time, such as the changes in the transportation networks and the decline of wetlands. It can be labor-intensive for a scientist to analyze changes across space and time in such maps, even after they have been digitized and converted to a vector format. In this paper, we present an unsupervised approach that converts vector data of geographic features extracted from multiple historical maps into linked spatio-temporal data. The resulting graphs can be easily queried and visualized to understand the changes in specific regions over time. We evaluate our technique on railroad network data extracted from USGS historical topographic maps for several regions over multiple map sheets and demonstrate how the automatically constructed linked geospatial data enables effective querying of the changes over different time periods.",
isbn="978-3-030-49461-2"
}

