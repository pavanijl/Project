@article{10.1145/3161602,
author = {Atluri, Gowtham and Karpatne, Anuj and Kumar, Vipin},
title = {Spatio-Temporal Data Mining: A Survey of Problems and Methods},
year = {2018},
issue_date = {September 2018},
publisher = {Association for Computing Machinery},
address = {New York, NY, USA},
volume = {51},
number = {4},
issn = {0360-0300},
url = {https://doi.org/10.1145/3161602},
doi = {10.1145/3161602},
abstract = {Large volumes of spatio-temporal data are increasingly collected and studied in diverse domains, including climate science, social sciences, neuroscience, epidemiology, transportation, mobile health, and Earth sciences. Spatio-temporal data differ from relational data for which computational approaches are developed in the data-mining community for multiple decades in that both spatial and temporal attributes are available in addition to the actual measurements/attributes. The presence of these attributes introduces additional challenges that needs to be dealt with. Approaches for mining spatio-temporal data have been studied for over a decade in the data-mining community. In this article, we present a broad survey of this relatively young field of spatio-temporal data mining. We discuss different types of spatio-temporal data and the relevant data-mining questions that arise in the context of analyzing each of these datasets. Based on the nature of the data-mining problem studied, we classify literature on spatio-temporal data mining into six major categories: clustering, predictive learning, change detection, frequent pattern mining, anomaly detection, and relationship mining. We discuss the various forms of spatio-temporal data-mining problems in each of these categories.},
journal = {ACM Comput. Surv.},
month = aug,
articleno = {83},
numpages = {41},
keywords = {Data mining, spatial data, geographic data, rasters, time-series, spatiotemporal data}
}