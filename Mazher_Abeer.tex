
@Article{w12020590,
AUTHOR = {Mazher, Abeer},
TITLE = {Visualization Framework for High-Dimensional Spatio-Temporal Hydrological Gridded Datasets using Machine-Learning Techniques},
JOURNAL = {Water},
VOLUME = {12},
YEAR = {2020},
NUMBER = {2},
ARTICLE-NUMBER = {590},
URL = {https://www.mdpi.com/2073-4441/12/2/590},
ISSN = {2073-4441},
ABSTRACT = {Numerical modelling increasingly generates massive, high-dimensional spatio-temporal datasets. Exploring such datasets relies on effective visualization. This study presents a generic workflow to (i) project high-dimensional spatio-temporal data on a two-dimensional (2D) plane accurately (ii) compare dimensionality reduction techniques (DRTs) in terms of resolution and computational efficiency (iii) represent 2D projection spatially using a 2D perceptually uniform background color map. Machine learning (ML) based DRTs for data visualization i.e., principal component analysis (PCA), generative topographic mapping (GTM), t-distributed stochastic neighbor embedding (t-SNE) and uniform manifold approximation and projection (UMAP) are compared in terms of accuracy, resolution and computational efficiency to handle massive datasets. The accuracy of visualization is evaluated using a quality metric based on a co-ranking framework. The workflow is applied to an output of an Australian Water Resource Assessment (AWRA) model for Tasmania, Australia. The dataset consists of daily time series of nine components of the water balance at a 5 km grid cell resolution for the year 2017. The case study shows that PCA allows rapid visualization of global data structures, while t-SNE and UMAP allows more accurate representation of local trends. Furthermore, UMAP is computationally more efficient than t-SNE and least affected by the outliers compared to GTM.},
DOI = {10.3390/w12020590}
}



