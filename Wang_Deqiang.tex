@inproceedings{10.1145/3152465.3152473,
author = {Wang, Deqiang and Guo, Danhuai and Zhang, Hui},
title = {Spatial Temporal Data Visualization In Emergency Management: A View from Data-Driven Decision},
year = {2017},
isbn = {9781450354936},
publisher = {Association for Computing Machinery},
address = {New York, NY, USA},
url = {https://doi.org/10.1145/3152465.3152473},
doi = {10.1145/3152465.3152473},
abstract = {Recent years, extreme events caused a great loss of human society. Emergency management is playing a more and more important role in handling disaster events. With the raising of data-intensive decision making, how to visualize large, multi-dimension data become an important challenge. Spatial temporal data visualization, a powerful tool, could transform data in to visual structure and make core information easily be captured by human. It could support spatial analysis, decision making and be used in all phase of emergency management. In this paper, we reviewed the general method of spatial temporal data visualization and the methods in data-intensive environment. Summarized the problems of each phase of emergency management and presented how spatial temporal visualization tools applied in each phase of emergency management. Finally, we conduct a short conclusion and outlook the future of spatial temporal visualization applied in data-driven emergency management environment.},
booktitle = {Proceedings of the 3rd ACM SIGSPATIAL Workshop on Emergency Management Using},
articleno = {8},
numpages = {7},
keywords = {emergency management, review, spatio-temporal visualization},
location = {Redondo Beach, CA, USA},
series = {EM-GIS'17}
}