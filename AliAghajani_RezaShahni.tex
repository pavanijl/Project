@article{AGHAJANI20172126,
title = {Applying GIS to Identify the Spatial and Temporal Patterns of Road Accidents Using Spatial Statistics (case study: Ilam Province, Iran)},
journal = {Transportation Research Procedia},
volume = {25},
pages = {2126-2138},
year = {2017},
note = {World Conference on Transport Research - WCTR 2016 Shanghai. 10-15 July 2016},
issn = {2352-1465},
doi = {https://doi.org/10.1016/j.trpro.2017.05.409},
url = {https://www.sciencedirect.com/science/article/pii/S2352146517307160},
author = {Mohammad Ali Aghajani and Reza Shahni Dezfoulian and Abdolreza Rezaee Arjroody and Mohammadreza Rezaei},
keywords = {GIS, Spatial statistic, Temporal Patterns, Road Accidents, Ilam Province, Iran},
abstract = {The number of fatalities and casualties caused by road accidents is mostly affected by the 3 factors of road, human, and vehicle. Nowadays, tackling with the accident prone locations including the definition, identification, and modification prioritization has attracted attention as an approach to enhance and improve the roads network safety level. A road accident analysis method is through the use of the Geographic Information System (GIS) and spatial and temporal patterns in accident prone locations. Since accidents are temporal phenomena, in this paper, GIS-based spatial statistical methods have been used to identify and model accident hot spots; in other words, we have investigated the use of localization patterns and hot spot distribution with the help of temporal information.Hot spot analysis with identification and data generation helps decision makers to take appropriate measures to decrease road accidents. To specify and analyze accidents distribution, the information regarding the accidents in the roads of Ilam Province (Iran 2013), has been investigated. Information included the accident type (fatality, injury).From the hot spot map, it is concluded that in northwest roads, despite less traffic, the number (spatial weight) of fatalities is more.This can be due to such factors as the route geometrical design, lack of appropriate relief, and so on. To identify the temporal patterns and accidents distribution, and also analyze the hot spots, Moran's method of spatial autocorrelation and Getis-OrdGi* statistic have been used.}
}