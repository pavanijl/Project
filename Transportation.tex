@article{KWAN2000185,
title = {Interactive geovisualization of activity-travel patterns using three-dimensional geographical information systems: a methodological exploration with a large data set},
journal = {Transportation Research Part C: Emerging Technologies},
volume = {8},
number = {1},
pages = {185-203},
year = {2000},
issn = {0968-090X},
doi = {https://doi.org/10.1016/S0968-090X(00)00017-6},
url = {https://www.sciencedirect.com/science/article/pii/S0968090X00000176},
author = {Mei-Po Kwan},
abstract = {A major difficulty in the analysis of disaggregate activity-travel behavior in the past arises from the many interacting dimensions involved (e.g. location, timing, duration and sequencing of trips and activities). Often, the researcher is forced to decompose activity-travel patterns into their component dimensions and focus only on one or two dimensions at a time, or to treat them as a multidimensional whole using multivariate methods to derive generalized activity-travel patterns. This paper describes several GIS-based three-dimensional (3D) geovisualization methods for dealing with the spatial and temporal dimensions of human activity-travel patterns at the same time while avoiding the interpretative complexity of multivariate pattern generalization or recognition methods. These methods are operationalized using interactive 3D GIS techniques and a travel diary data set collected in the Portland (Oregon) metropolitan region. The study demonstrates several advantages in using these methods. First, significance of the temporal dimension and its interaction with the spatial dimension in structuring the daily space-time trajectories of individuals can be clearly revealed. Second, they are effective tools for the exploratory analysis of activity diary data that can lead to more focused analysis in later stages of a study. They can also help the formulation of more realistic computational or behavioral travel models.}
}