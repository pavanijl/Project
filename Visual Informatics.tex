@article{CHEN2019129,
title = "A Survey of Multi-Space Techniques in Spatio-Temporal Simulation Data Visualization",
journal = "Visual Informatics",
volume = "3",
number = "3",
pages = "129 - 139",
year = "2019",
issn = "2468-502X",
doi = "https://doi.org/10.1016/j.visinf.2019.08.002",
url = "http://www.sciencedirect.com/science/article/pii/S2468502X19300452",
author = "Xueyi Chen and Liming Shen and Ziqi Sha and Richen Liu and Siming Chen and Genlin Ji and Chao Tan",
keywords = "Simulation data visualization, Spatio-temporal data visualization, Comparative visualization",
abstract = "The widespread use of numerical simulations in different scientific domains provides a variety of research opportunities. They often output a great deal of spatio-temporal simulation data, which are traditionally characterized as single-run, multi-run, multi-variate, multi-modal and multi-dimensional. From the perspective of data exploration and analysis, we noticed that many works focusing on spatio-temporal simulation data often share similar exploration techniques, for example, the exploration schemes designed in simulation space, parameter space, feature space and combinations of them. However, it lacks a survey to have a systematic overview of the essential commonalities shared by those works. In this survey, we take a novel multi-space perspective to categorize the state-of-the-art works into three major categories. Specifically, the works are characterized as using similar techniques such as visual designs in simulation space (e.g, visual mapping, boxplot-based visual summarization, etc.), parameter space analysis (e.g, visual steering, parameter space projection, etc.) and data processing in feature space (e.g, feature definition and extraction, sampling, reduction and clustering of simulation data, etc.)."
}